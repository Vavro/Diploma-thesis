\chapter{Analysis}

\section{Graph database speed}

\begin{itemize}
	\item Query speed - optimized
	\item Data retrieval speed - data locality - cant be optimized
	\item Options for faster load speed (simple results cache, document db)
	\item Conclusion choose a document database
\end{itemize}	
%%maybe move to analysis just say here it could be quicker.
Opposite to the graph database is usually considered to be document databases. Document databases consider the smallest entity to be the Domain aggregate which is stored as a whole document and always read at once. The biggest difference between these two database structures are the possibilities to query data
\begin{itemize}
	\item Graph databases provide ways to make complex queries over the whole data
	\item Document databases usually provide only ways to query one document (data entity)
\end{itemize}
Document databases are nowadays considered to be the goto database for quick data retrieval.



\section{Requirements for queries}
\begin{itemize}
	\item Get Detail
	\item Search document query operators
			
\end{itemize}

\section{Choosing a storage engine}
- Document database comparisons
\begin{itemize}
	\item Document database options
\end{itemize}

\section{Document storage options}
\begin{itemize}
	\item How to structure the data (JSON-LD)
		\subitem JSON-LD algorithms
		\subitem As whole document
			\subsubitem Flat
			\subsubitem Expanded
		\subitem Split to multiple documents
			\subsubitem How to split
\end{itemize}

\section{Usability possibilities}
\begin{itemize}
	\item How easy it could be to use
	\item 
\end{itemize}

\section{Optimalization - maybe to Conclusion}
\begin{itemize}
	\item Offload workload to client
		\subitem Transform of data (required storage implementation details) - i.e. have to rename the properties
		\subitem Analysis of sent queries (renamed properties of documents force a rename in the query)
\end{itemize}





